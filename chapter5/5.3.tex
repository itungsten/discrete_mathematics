%! Tex program = xelatex
\documentclass[UTF8]{article}
\usepackage{indentfirst}
\usepackage{graphicx} 
\usepackage{amsmath}  
\usepackage{float}   
\usepackage{listings}

\title{Discrete Mathematics}
\author{Zhengren Wang 2019081308021}
\date{05/05/2020 Tue }
\begin{document}
\maketitle 

\part{5.3}
\begin{description}
    \item[29]Give a recursive definition of each of these sets of ordered pairs of positive integers. Use structural induction to prove that the recursive definition you found is correct. \\\\
        a)\\
        \textbf{Definition}:\\
        Basis Step:$(1,1) \in S$  \\
        Recursive Step:if $(x,y)\in S$ ,then $(x+1,y+1),(x+2,y),(x,y+2) \in S$  \\


        \textbf{Proof}: \\ 
        Assum the proper set to be U.\\
        $S \subset U$ \\
        the origin sum of $(x,y)$ is 2, and the extended sum is always two greater than the before one. So all the sum of $(x,y)$ is even, and $S \subset U $. \\\\
        $U \subset S$  \\
        $\forall \;(x,y) \in U$ whose sum is $2k+2$, $k \geq 0$. \\
        If x is even and y is even, we can substract x or y by 2 to make x=y. \\
        Similarly, it the same with x is odd and y is odd. And there is no other case. \\
        When x=y, we can substract x and y by 1, until x=y=1. \\
        So, we find a path from (1,1) to an arbitrary (x,y), so $U \subset S$. \\\\
        Then  $U=S$ , as we desired. \\\\

        b)\\
        \textbf{Definition}:\\
        Basis Step:$(1,1),(1,2),(2,1) \in S$  \\
        Recursive Step:if $(x,y)\in S$ ,then $(x+2,y),(x,y+2) \in S$  \\\\

        \textbf{Proof}: \\ 
        Assum the proper set to be U.\\
        $S \subset U$ \\
         It really obvious that (1,1),(1,2) and (2,1) all have an odd coordinate at least. So all the derived $(x,y)$ have at least an odd coordinate. \\\\
        $U \subset S$ \\
         Conversely, while x>2 x-=2, while y>2 y-=2. After these steps, (a,b) can transformed into the basis (1,1),(1,2),(2,1), and they are in S.\\
        So, we find a path from the origins to (x,y), so $U \subset S$. \\\\
        Then  $U=S$ , as we desired. \\\\

        c)\\
        \textbf{Definition}:\\
        Basis Step:$(1,6),(2,3) \in S$  \\
        Recursive Step: If $(x,y)\in S$, then $(x+2,y)\in S$ and $(x,y+6)\in S$.\\

        \textbf{Proof}: \\ 
        Assum the proper set to be U.\\
        $S \subset U$ \\
        3|6 and 1+6=7 is odd, 3|3 and 2+3=5 is odd. If (x,y) is in S, then (x+2,b) and (x,y+6) which are supposed to be in S, are also in U.\\\\
        $U \subset S$ \\
        While x>2 x:=x-2, while y>6 y:=y-6. At last you will fall into (2,3) or (1,6).\\

        So, we find a path from the origins to (x,y), so $U \subset S$. \\\\
        Then  $U=S$ , as we desired. \\\\

\end{description}

\end{document}
