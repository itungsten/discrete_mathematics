%! Tex program = xelatex
\documentclass[UTF8]{article}
\usepackage{indentfirst}
\usepackage{graphicx} 
\usepackage{amsmath}  
\usepackage{float}   
\usepackage{listings}

\title{Discrete Mathematics}
\author{Zhengren Wang 2019081308021}
\date{05/05/2020 Tue }
\begin{document}
\maketitle 

\part{5.1}
\begin{description}
    \item[5]Prove that 
        $$1^2 + 3^2 + 5^2 + \cdots + (2n + 1)^2 = \frac{(n + 1)(2n + 1)(2n + 3)}{3}$$ 
\quad \quad whenever $n$ is a nonnegative integer \\\\
        Let $P(n)$ to be 
        $$1^2 + 3^2 + 5^2 + \cdots + (2n + 1)^2 = \frac{(n + 1)(2n + 1)(2n + 3)}{3}$$ 
        Basis Step: In terms of $P(0)$, $1^2=\frac{(0+1)(0+1)(0+3)}{3}=1$ is True \\\\
        Inductive Step:Assum $P(k-1)$ is True, now we prove $P(k)$ is True \\
$$1^2 + 3^2 + 5^2 + \cdots + (2k + 1)^2 =1^2 + 3^2 + 5^2 + \cdots + (2k-1)^2+ (2k + 1)^2 = \frac{k(2k - 1)(2k + 1)}{3}+(2k+1)^2$$  \\
        Meanwhile, we have \quad $$\frac{k(2k - 1)(2k + 1)}{3}+(2k+1)^2=\frac{(2k^2+5k+3)(2k + 1)}{3}=\frac{(k + 1)(2k + 1)(2k + 3)}{3}$$\\
        which implies $P(k)$ is True. \\
        So, in the end, we have 
                $$1^2 + 3^2 + 5^2 + \cdots + (2k + 1)^2 = \frac{(k + 1)(2k + 1)(2k + 3)}{3}$$ 
        as desired. \\

    \item[51]What is wrong with this "proof"? 
    "Theorem":For every positive integer n, if x and y are positive integers with $max(x, y)$ = n, then x = y.  
    Basis Step: Suppose that n = 1. If $max(x, y)$ = 1 and x and y are positive integers, we have x = 1 and y = 1. 
    Inductive Step: Let k be a positive integer. Assume that whenever $max(x, y)$ = k and x and y are positive integers, then x = y. Now let max(x, y) = k + 1, where x and y are positive integers. Then $max(x-1, y-1)$ = k, so by the inductive hypothesis, $x-1$ = $y-1$. It follows that x = y, completing the inductive step. \\\\

    The mistake is in the Inductive Step. When we have x,y to be positive, we cannot say $x-1,y-1$ is also positive.

\end{description}

\end{document}
