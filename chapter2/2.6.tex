%! Tex program = xelatex
\documentclass[UTF8]{article}
\usepackage{indentfirst}
\usepackage{graphicx} 
\usepackage{amsmath}  
\usepackage{float}   
\usepackage{listings}

\title{Discrete Mathematics}
\author{Zhengren Wang 2019081308021}
\date{04/24/2020 Fri }
\begin{document}
\maketitle 

\part{2.6}
\begin{description}
    \item[15] Let
        \begin{equation} A=
            \left[
                \begin{array}{cc}
                    1 & 1 \\
                    0 & 1 \\
                \end{array}
            \right]
        \end{equation} Find a formula for $A^n$, whenever n is a positive integer.\\\\
        \begin{equation}
            A^{n}=
            \left[
                \begin{array}{cc}
                    1 & n\\
                    0 & 1\\
                \end{array}
            \right]
        \end{equation}
        \item[29]Let
		\begin{equation}
            A=
			\left[
			\begin{array}{ccc}
			1 & 0 & 0\\
			1 & 0 & 1\\
			0 & 1 & 0
			\end{array}
			\right]
		\end{equation}
		Find\\
		a) $A^{[2]}$\\
		b) $A^{[3]}$\\
		c) $A\lor A^{[2]}\lor A^{[3]}$\\\\
		a) 
		\begin{equation}
		A^{[2]}=
		\left[
		\begin{array}{ccc}
		1 & 0 & 0\\
		1 & 1 & 0\\
		1 & 0 & 1
		\end{array}
		\right]
		\end{equation}
		b) 
		\begin{equation}
		A^{[3]}=
		\left[
		\begin{array}{ccc}
		1 & 0 & 0\\
		1 & 0 & 1\\
		1 & 1 & 0
		\end{array}
		\right]
		\end{equation}
		c) 
		\begin{equation}
		A\lor A^{[2]}\lor A^{[3]}=
		\left[
		\begin{array}{ccc}
		1 & 0 & 0\\
		1 & 1 & 1\\
		1 & 1 & 1
		\end{array}
		\right]
		\end{equation}
        

\end{description}

\end{document}
