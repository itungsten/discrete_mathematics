%! Tex program = xelatex
\documentclass[UTF8]{article}
\usepackage{indentfirst}
\usepackage{graphicx} 
\usepackage{amsmath}  
\usepackage{float}   
\usepackage{listings}

\title{Discrete Mathematics}
\author{Zhengren Wang 2019081308021}
\date{05/28/2020 Thu }
\begin{document}
\maketitle 

\part{9.1}
\begin{description}
    \item[3]For each of these relations on the set \{1, 2, 3, 4\}, decide whether it is reflexive, whether it is symmetric, whether it is antisymmetric, and whether it is transitive. \\
            a) \{(2, 2), (2, 3), (2, 4), (3, 2), (3, 3), (3, 4)\}	\\
            b) \{(1, 1), (1, 2), (2, 1), (2, 2), (3, 3), (4, 4)\} \\
            c) \{(2, 4), (4, 2)\}	\\
            d) \{(1, 2), (2, 3), (3, 4)\}	\\
            e) \{(1, 1), (2, 2), (3, 3),(4, 4)\}	\\
            f) \{(1, 3), (1, 4), (2, 3), (2, 4), (3, 1), (3, 4)\}	\\

            a) transitive                                   \\
            b) reflexive,symmetric,transitive               \\
            c) symmetric                                    \\
            d) antisymmetric                                \\
            e) reflexive,symmetric,antisymmetric,transitive \\
            f) none                                         \\


    \item[41]Let R1 and R2 be the "congruent modulo 3" and the "congruent modulo 4" relations, respectively, on the set of integers. That is, $R1 = {(a, b) | a \equiv b (mod 3)}$ and $R2 = {(a, b) | a \equiv b (mod 4)}$. Find  \\
             a) $R1 \cup R2. $   \\
             b) $R1 \cap R2. $  \\
             c) $R1 - R2. $  \\
             d) $R2 - R1. $  \\
             e) $R1 \oplus R2. $  \\

             a)$\{(a,b) \;| a-b \;\equiv 0,3,4,6,8,9 \;(mod 12)\}$   \\
             b)$\{(a,b) \;| a   \;\equiv b              \;(mod 12)\}$   \\
             c)$\{(a,b) \;| a-b \;\equiv 3,6,9       \;(mod 12)\}$   \\
             d)$\{(a,b) \;| a-b \;\equiv 4,8         \;(mod 12)\}$   \\
             e)$\{(a,b) \;| a-b \;\equiv 3,4,6,8,9   \;(mod 12)\}$   \\


\end{description}

\end{document}
