%! Tex program = xelatex
\documentclass[UTF8]{article}
\usepackage{indentfirst}
\usepackage{graphicx} 
\usepackage{amsmath}  
\usepackage{float}   
\usepackage{listings}

\title{Discrete Mathematics}
\author{Zhengren Wang 2019081308021}
\date{05/28/2020 Thu }
\begin{document}
\maketitle 

\part{9.2}
\begin{description}
    \item[7]The 3-tuples in a 3-ary relation represent the following attributes of a student database: student ID number, name, phone number. \\
        a) Is student ID number likely to be a primary key?              \\
        b) Is name likely to be a primary key?                           \\
        c) Is phone number likely to be a primary key?                   \\

        a) Yes. Student ID number is likely to be a primary key.            \\
        b) No. Name isn't likely to be a primary key.                      \\
        c) No. Phone number isn't likely to be a primary key.                 \\
    \item[11]What do you obtain when you apply the selection operator $s_C$, where $C$ is the condition $Destination = Detroit$, to the database in Table 8? \\
            (Nadir, 122, 34, Detroit, 08 : 10)      \\
            (Nadir, 199, 13, Detroit, 08 : 47)      \\
            (Nadir, 322, 34, Detroit, 09 : 44)      \\

\end{description}

\end{document}
