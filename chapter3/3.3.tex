%! Tex program = xelatex
\documentclass[UTF8]{article}
\usepackage{indentfirst}
\usepackage{graphicx} 
\usepackage{amsmath}  
\usepackage{float}   
\usepackage{listings}

\title{Discrete Mathematics}
\author{Zhengren Wang 2019081308021}
\date{04/24/2020 Fri }
\begin{document}
\maketitle 

\part{3.3}
\begin{description}
    \item[3]Give a $O$ estimate for the number of operations,
where an operation is a comparison or a multiplication,
used in this segment of an algorithm(ignoring comparisons used to test the conditions in the for loops, where
a1, a2, ..., an are positive real numbers). \\\\
$O{(n^2)}$ \\
    \item[7]Suppose that an element is known to be among the first four elements in a list of 32 elements. Would a linear search or a binary search locate this element more rapidly \\\\
        \textbf{linear} search
    \item[11]
        a)Suppose we have n subsets S1, S2,...,Sn of the set {1, 2,...,n}. Express a brute-force algorithm that determines whether there is a disjoint pair of these subsets  \\\\
\hspace*{1cm}    \textbf{procedure} disjointpair($S_1, S_2,\cdots,S_n$ : subsets of $\{1,2,\cdots,n\}$)  \\
\hspace*{1cm}    \textbf{for} i := 1 to n   \\
\hspace*{2cm}       \textbf{for} j := i+1 to n  \\
\hspace*{3cm}            m:=1 \\
\hspace*{3cm}            \textbf{for} k :=1 to n  \\
\hspace*{4cm}                \textbf{if}( $k \notin S1 \;\land \; k \notin S2$), \textbf{then}  \\
\hspace*{5cm}                    m:=0   \\
\hspace*{5cm}                    \textbf{break}   \\
\hspace*{3cm}            if(m=1)   \\
\hspace*{4cm}                \textbf{return} True   \\
\hspace*{1cm}    \textbf{return} False   \\\\

        b)Give a big-O estimate for the number of times the algorithm needs to determine whether an integer is in one of the subsets. \\\\
        $O{(n^3)}$ \\
        \item[15]What is the largest n for which one can solve within one
second a problem using an algorithm that requires f (n)
bit operations, where each bit operation is carried out in
10−9 seconds, with these functions f (n)?  \\\\
a) $2^{10^{9}}$                                  \\
b) $10^{9}$                                      \\
c) $3.96\times 10^7$\\
d) $\lfloor 10^{4.5} \rfloor$  \quad  i.e. \; 31622            \\
e) $9\times log_2(10)$  \quad  i.e. \; 29                     \\
f)  12                                           \\
        
\end{description}

\end{document}
